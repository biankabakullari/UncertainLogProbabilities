In this chapter, we obtain probability estimates for the trace realizations of uncertain instances by using the information that the rest of the traces from the log provide.
In contrast to the way we obtained these estimates in the last chapter, the information we exploit here is not the one encoded within each event describing uncertainty explicitly. 
Instead, the new estimate solely quantifies how likely it is to see other traces in the log that are identical or display similar patterns to the current one.
The idea and the proposed models in this section are inspired from the work done in \cite{por}.
The main motivation in \cite{por} is obtaining better conformance assessments for process instances whose events can only be partially ordered.
This happens when the recorded timestamps of some events do not allow a unique (total) order between them.
%Possible causes could be lack of synchronization, manual recordings and so on.
In their work, the authors define such events as \textit{uncertain events} and the lack of a total order on them as \textit{order uncertainty}.

An important assumption affecting the methods used in \cite{por} is that any two events which can not be totally ordered have identical timestamps.
This enables partitioning any event set $E$ belonging to some proces instance from the log into non-empty subsets $E=\langle E_1,...,E_m \rangle $ such that:
\begin{itemize}
\item \textbf{(1)} Any pair of events from different subsets can be totally ordered.
More precisely, for all $1 \leq i < j \leq m$: each event in $E_i$ happened before each event from $E_j$.  
\item \textbf{(2)} All events in the same subset have identical timestamps and therefore every permutation on them is a possible ordering.
\end{itemize}
Certain traces correspond to the special case where $|E_i| = 1$ for all $1 \leq i \leq m$.

Since our definition of uncertainty is broader, the assumption above does not necessarily hold for all uncertain cases.
Given some event set $E$, we can define the subsets of $E$ to be the subsets of vertices in the connected components $(CC(\mathcal{I}))$ of the interval graph $\mathcal{I}(E)$.
By Theorem \ref{theorem: partitioning}, we know that all pairs of elements across subsets can be totally ordered and we can sort the events in the components into a sequence $\langle E_1,...,E_{|CC(I)|}\rangle$ in such a way that they satisfy condition \textbf{(1)}.
Condition \textbf{(2)} on the other hand holds if and only if all the connected components in the interval graph induce cliques (see Proposition \ref{prop: clique}).
This is however not always the case.

In the following, we adapt the methods introduced in \cite{por} to obtain new probability estimates for the trace realizations based on behavioral regularities in the log.
There are three different approaches, each an estimator that incorporates different notions of behavioral abstraction.
In other words, each method gives a different view on what patterns to look for in the other traces from the log, in order to measure the likelihood of a particular activity trace realization.
Given a particular activity trace, the level of abstraction of the chosen method determines how similar other traces in the log have to be in order to consider the similarity meaningful and take it into account when measuring the likelihood of the trace.


\section{Probability Estimation Models}
From now on, suppose that we are given a log $L$ and let $\mathcal{U}_C^L$ be the set of all case IDs appearing in $L$.
For any $c \in \mathcal{U}_C^L$, let $E_c \subseteq L$ be the maximal set of events belonging to case $c$.
%We define $CertainCases := \{c \in \mathcal{U}_C(L) \mid 
%\forall e \in E_c: e \in \mathcal{E}_C\}$
%to be the set of certain cases, that is those cases whose events are all certain.
%The set of uncertain cases is defined as
%$UncertainCases := \mathcal{U}_C(L) \setminus CertainCases$.
For every case $c \in \mathcal{U}_C^L$ and its event set $E_c$, let $\mathcal{F}(E_c)$ be its follows graph and $\mathcal{I}(E_c)$ be its interval graph.
By sorting the connected components of the interval graph as we explained above, we can always partition $E_c$ into non-empty subsets and order them into a sequence $\langle E_1,...,E_m\rangle$ with $m=|CC(\mathcal{I})|$ such that they satisfy condition \textbf{(1)}.
We use $\lambda(E_c)$ to identify the ordered partition of event set $E_c$.
Note that if one can totally order the events from $E_c$, the subsets in the ordered partitions each have size 1.
Even events with uncertain timestamps might be totally ordered by definition of $\prec_{\mathcal{E}}$, as long as their time intervals do not overlap.
%Additionally, for any certain case $c \in CertainCases$, we use $\sigma_e(c)$ and $\sigma_a(c)$ to identify its unique event and activity trace respectively.


\subsection{Trace Equivalence Model}
The \textit{trace equivalence model} estimates the probability of a sequence of activities by exploring how often the particular trace is the only possile activity trace of cases with a unique event sequence.
Given some possible activity trace $s_a$ of an uncertain case, we measure its probability as:
\begin{align*}
\textbf{p}_{trace}(s_a) = \eta \cdot \frac{|\{ c \in \mathcal{U}_C^L \mid \mathcal{R}_e(E_c)= \{s_e\} ~ \wedge A(s_e)= \{s_a \} \} }
{|\{ c \in \mathcal{U}_C^L \mid \mathcal{R}_e(E_c)= \{s_e\} ~ \wedge ~ |A(s_e)| = 1\}|}
\end{align*}
where $\eta$ is the normalizing factor over all activity traces of the uncertain case.\\
This way, the probability value that $P_{trace}$ assigns to an activity trace yields the fraction of traces with totally ordered event set and no uncertainty in activities that are equivalent to it.
In contrast to the model definition in \cite{por}, we do not only look at traces containing only certain events, but also include instances with uncertainty in timestamps as long as their event set can still be totally ordered.
While this model is rather simple, it may have limited applicability if there are very few such traces present in the log.
Because of the low abstraction level, it might also be that one can hardly find equivalent traces even if the number in the denominator is large.


\subsection{N-gram Model}
Instead of considering only fully equivalent traces, in this model the authors in \cite{por} explore how often subsequences appear in the log.
Given an activity trace, first it is measured how probable it is for each activity in the trace to appear at the corresponding position.
Here, the authors measure how often each activity label in the trace succeeds its up to the last $N-1$ preceeding activities.
This estimation is, however, only based on all traces of the log that contain the relevant activities in the subsequence without order uncertainty.
For this, a predicate $certain$ is defined the following way: 
Given an event set $E$ with $\lambda(E)=\langle E_1,...,E_m\rangle$ and a subsequence $\langle a_1,...,a_l\rangle$ of a possible activity trace $s_a$, we define
\begin{align*}
certain(\langle a_1,...,a_l \rangle, \langle E_1,...,E_m \rangle) \Leftrightarrow \\
\exists ~ i \in \{0,...,m-l\} ~ s.t.~
\forall ~ j \in \{1,...,l\}: E_{i+j} = \{e\} \text{ for some } e \in E \\ \wedge ~ \pi_o(e)=! \wedge ~ \pi_a(e)= a_j.
\end{align*}
This way, a subsequence of activities is considered as \textit{certain} in an event set if there is a subsequence of determinate events which can be certainly ordered in time that execute the activities in the given order.
In contrast to the definition used in \cite{por}, here we also require the events in the subsequence to be determinate.
The predicate helps measure the probability of a particular activity label succeeding its predecessors:
\begin{align*}
P(a ~ | ~ \langle a_1,...,a_l \rangle) = \frac{|\{c \in \mathcal{U}_C^L \mid
certain(\langle a_1,...,a_l,a \rangle, \lambda(E_c)) \}|}
{|\{c \in \mathcal{U}_C^L \mid
certain(\langle a_1,...,a_l \rangle, \lambda(E_c)) \}|}.
\end{align*}
Given a possible activity trace $s_a=\langle a_1,...,a_n\rangle$, its probability is obtained by aggregating the probability of each label succeeding its up to $N-1$ preceeding activities in $s_a$:
\begin{align*}
\textbf{p}_{\text{\textit{N-gram}}}(\langle a_1,...,a_n \rangle) =  \eta \cdot
\prod_{k=2}^n P(a_k ~ | ~ \langle a_{max(1,k-N+1)},..., a_{k-1}\rangle)
\end{align*}
where $\eta$ is the normalizing factor over all activity traces of the uncertain case.\\
As explained in \cite{por}, the approach may be adapted to explicitly consider the first events of traces in the assessment by adding a new artificial activity label to the first position of all activity traces in the log.
The authors argue that this model is more abstract compared to the trace equivalence model.
Indeed, it only requires finding existing traces in the log that have equivalent subsequences without uncertainty, instead of fully equivalent certain traces.
This is useful if the process has local dependencies which are independent from other parts of the execution.
Think of a process where the first three activities are always executed in the same order.
Given two activity trace realizations which only differ in the ordering of these first three activities, one might want to consider the trace in which those first three activities have the expected order more likely, even if the rest of both traces find no resemblance in the log.

The parameter $N$, however, determines the level of abstraction.
If $N$ is at least as large as the longest trace in the log, the \textit{N-gram} and the \textit{trace equivalence} model are equivalent.
A drawback is also that in the N-gram model, the behavioral regularities needed to support an uncertain trace always have the form of consecutive activities.
There could be activities whose dependency lies in the fact that they happen in a particular order, but not necessarily consecutively.
This is the motivation behind the third model which we introduce next.



\subsection{Weak Order Model}
The \textit{weak order model} proposed in \cite{por} takes all order dependencies into account, even if they do not materialize in the form of consecutive activity executions.
Such activities have an indirect order dependency, a \textit{weak order}, from which the method gets its name.
First, a predicate $order$ is defined.
Given two activity labels $a,a'$ and the event set of a process instance from the log, we have:
\begin{align*}
order(a,a',E) \Leftrightarrow 
\exists ~e, e' \in E: ~ e \prec_{\mathcal{E}} e' ~ \wedge \pi_o(e)=! ~ \wedge \pi_o(e')=! \\
\wedge ~ \pi_a(e)=a \wedge \pi_a(e')=a'.
\end{align*}

$Order$
captures whether two particular (and not necessarily distinct) activities appear in a particular order in a given process instance.
Regardless of whether the case in question is certain or uncertain, the predicate is satisfied if there exist two determinate events that can be totally ordered in time, where the first event executes the first activity and the second one the second activity.
Again, in contrast to the definition used in \cite{por}, here we also require the pair of events to be determinate.
Note that equivalent to verifying whether the events satisfy the $\prec_{\mathcal{E}}$ relation, one could test if there is an arc between them in the follows graph $\mathcal{F}(E)$ (Def. \ref{def: follows graph}).
The predicate helps assessing how often events execute a pair of particular activities in a given order when considering only the traces that certainly contain both activities:
\begin{align*}
P(a,a')= \frac{|\{c \in \mathcal{U}_C^L \mid order(a,a',E_c)\}|}
{|\{c \in \mathcal{U}_C^L \mid \exists ~ e,e' \in E_c: ~
\pi_o(e)=\pi_o(e')=! \wedge \pi_a(e)=a \wedge \pi_a(e')=a'\}|}.
\end{align*}
From here on, one estimates the probability of a possible activity trace $s_a=\langle a_1,...,a_n \rangle$ by aggregating the probabilities that each pair of activities in $s_a$ has the order indicated in $s_a$:
\begin{align*}
\textbf{p}_{WO}(\langle a_1,...,a_n \rangle) = \eta \cdot \prod_{
\substack{1 \leq i < n \\ i < j \leq n}}
P(a_i,a_j)
\end{align*}
where $\eta$ is the normalizing factor over all activity traces of the uncertain case.
The weak order model uses the most abstract notion of behavioral regularities when deciding which similarities across traces are considered relevant.

Given an uncertain process instance with event set $E$ and set $\mathcal{R}_a$ containing all possible activity traces, we can compute a new expected conformance score the following way:
\begin{align*}
\overline{Conf}(E) = \sum_{s_a \in \mathcal{R}_a} \textbf{p}_m(s_a) \cdot conf(s_a,M)
\end{align*}
where $M$ is the process model, $conf(s_a,M)$ yields the conformance score of trace $s_a$ and model $M$, and $m \in \{trace, {\text{\textit{N-gram}}}, WO\}$ is the chosen method.

It is important to stress that in all three methods, the process model itself is not included when evaluating the probabilities.
This is crucial if the probability estimates are used as weights to compute conformance checking scores.
Otherwise, the model would introduce a bias in the conformance checking results, by for example increasing the weights of the conforming traces.

\section{Conformance Checking Combining Two Probability \\ Distributions}
Until now, we have seen how, for a given trace realization of an uncertain case, we can obtain two probability values: one computed using the information on uncertainty on the event level and one reflecting how similar the trace is to other traces in the log.
Naturally, we can exploit both estimates to aggregate a new probability for each trace.
However, this is useful only when the two estimates are computed based on independent information.
This way, the information on uncertainty enclosing the events of the uncertain case should not contain information reflecting the model or the behavioral regularities in the log.
Similarly, the probability estimate computed with the trace equivalence model, N-gram or the weak order model should rely only on patterns which appear without uncertainty.
As we saw in the definitions of the three models, uncertain trace realizations are also considered for estimating the probabilities based on trace patterns in the log.
However, such traces display the relevant pattern or behavioral regularity without uncertainty.
 
Assuming that the two probability estimates are independent, a new probability estimate for an activity trace $s_a$ is obtained the following way:
\begin{align*}
&\textbf{p}_{log}(s_a) \gets \textbf{p}_m(s_a) \text{ \hspace*{2cm} where } m \in \{trace,{\text{\textit{N-gram}}},WO\}, \\
&\textbf{p}_{unc}(s_a) \gets \textbf{p}(s_a) \text{\hspace*{2,3cm} as defined in Chapter \ref{chap:estimates}, and} \\ 
&\textbf{p}_{combi}(s_a) = w_{log} \cdot \textbf{p}_{log}(s_a) + w_{unc} \cdot \textbf{p}_{unc}(s_a),
\end{align*}
where $w_{log}, w_{unc}$ are two non-negative weights that sum up to 1.

The values of the weights $w_{log}$ and $w_{unc}$ can be chosen in a way that they reflect the desired focus or our reliability in the estimates and in the data.

%\textcolor{red}{Add small example with artificial log and use Pwo because it is simple but still abstract enough for a small example.}

\section{Obtaining Probability Estimates for Example Log}
Suppose we have an uncertain log $L$ containing 35 process instances, from which 29 are certain traces and 6 are uncertain.
We represent $L$ as a multiset of behavior graphs \cite{space} (see Fig. \ref{fig: combi estimates}) consisting of 6 distinct graphs.
The first three graphs show the three variants of the 29 certain traces.
Notice that these graphs are paths, there is no uncertainty in activities and all events are determinate.
The other three graphs are the behavior graphs of the 6 uncertain traces.
Suppose that the uncertain trace for which we want to estimate the probabilities is the one represented by the last graph having uncertainty type $[A]_{\mathbb{W}}[T]_{\mathbb{S}}$.
Since the two events in the middle can appear in any order, there are two possible event sequences each with probability $1/2$.
Since one of the events in the middle has two possible activity labels, this process instance has four possible activity traces:
$\langle a,b,d,e\rangle, \langle a,c,d,e\rangle, \langle a,d,b,e\rangle$ and $\langle a,d,c,e\rangle$.
As we can see in Table \ref{table: combi traces}, the probability $\textbf{p}_{unc}$ of each activity trace depends on whether the trace contains activity $b$ or activity $c$.
Next, we estimate the values of $\textbf{p}_{log}$ for these four traces using the weak order model.
This requires estimating the probability $P(a_1,a_2)$ for all activity pairs $a_1,a_2$ that appear in one of the four traces in this particular order.
These estimates can be obtained from Table \ref{table: combi pairs}.
These values are then aggregated to obtain the weak-order probability values as shown in the third column of Table \ref{table: combi traces}.
%
%
%
%
%
\begin{figure}[h]
	\centering
	\begin{tikzpicture}[->,>=stealth',shorten >=1pt,node 						distance=2.5cm,auto,main node/.style={circle,draw,align=center}]
	%\draw [help lines] (0,0) grid (10,15);
	
	%Graph with #12
	\node[] at (0,15) {\large 12x};
	\node[main node,label=above: \large $a$] (A1) at (1,15) {\textcolor{white}{$e$}};
	\node[main node,label=above: \large $b$] (B1) at (3,15) {\textcolor{white}{$e$}};
	\node[main node,label=above: \large $c$] (C1) at (5,15) {\textcolor{white}{$e$}};
	\node[main node,label=above: \large $d$] (D1) at (7,15) {\textcolor{white}{$e$}};
	\node[main node,label=above: \large $e$] (E1) at (9,15) {\textcolor{white}{$e$}};
	
	%Graph with #10
	\node[] at (0,13) {\large 10x};
	\node[main node,label=above: \large $a$] (A2) at (2,13) {\textcolor{white}{$e$}};
	\node[main node,label=above: \large $b$] (B2) at (4,13) {\textcolor{white}{$e$}};
	\node[main node,label=above: \large $c$] (C2) at (6,13) {\textcolor{white}{$e$}};
	\node[main node,label=above: \large $e$] (D2) at (8,13) {\textcolor{white}{$e$}};

	%Graph with #7
	\node[] at (0,11) {\large 7x};
	\node[main node,label=above: \large $a$] (A3) at (2,11) {\textcolor{white}{$e$}};
	\node[main node,label=above: \large $d$] (B3) at (4,11) {\textcolor{white}{$e$}};
	\node[main node,label=above: \large $c$] (C3) at (6,11) {\textcolor{white}{$e$}};
	\node[main node,label=above: \large $e$] (D3) at (8,11) {\textcolor{white}{$e$}};

	%Graph with #1
	\node[] at (0,9) {\large 1x};
	\node[main node,label=above: \large $a$] (A4) at (1,9) {\textcolor{white}{$e$}};
	\node[main node,label=above: \large $b$] (B4) at (3,9) {\textcolor{white}{$e$}};
	\node[main node,label=above: \large ${\{c,d\}}$] (C4) at (5,9) {\textcolor{white}{$e$}};
	\node[main node,label=above: \large $d$] (D4) at (7,9) {\textcolor{white}{$e$}};
	\node[main node,label=above: \large $e$] (E4) at (9,9) {\textcolor{white}{$e$}};
	
	%Graph with #3
	\node[] at (0,6) {\large 3x};
	\node[main node,label=above: \large $a$] (A5) at (2,6) {\textcolor{white}{$e$}};
	\node[main node,label=above: \large $b$] (B5) at (4,6) {\textcolor{white}{$e$}};
	\node[main node,label=above: \large $c$] (C5) at (6,7) {\textcolor{white}{$e$}};
	\node[main node,label=above: \large $d$] (D5) at (6,5) {\textcolor{white}{$e$}};
	\node[main node,label=above: \large $e$] (E5) at (8,6) {\textcolor{white}{$e$}};	
	
	%Graph to estimate
	\node[] at (0,2) {\large 1x};
	\node[main node,label=above: \large $a$] (A6) at (3,2) {\textcolor{white}{$e$}};
	\node[main node,label=above: \large ${\{b: 0.3, ~c: 0.7\}}$] (B6) at (5,3) {\textcolor{white}{$e$}};
	\node[main node,label=above: \large $d$] (C6) at (5,1) {\textcolor{white}{$e$}};
	\node[main node,label=above: \large $e$] (D6) at (7,2) {\textcolor{white}{$e$}};

	
	\path
	
	(A1) edge (B1)
	(B1) edge (C1)
	(C1) edge (D1)
	(D1) edge (E1)
	
	(A2) edge (B2)
	(B2) edge (C2)
	(C2) edge (D2)
	
	(A3) edge (B3)
	(B3) edge (C3)
	(C3) edge (D3)
	
	(A4) edge (B4)
	(B4) edge (C4)
	(C4) edge (D4)
	(D4) edge (E4)
	
	(A5) edge (B5)
	(B5) edge (C5)
	(B5) edge (D5)
	(C5) edge (E5)
	(D5) edge (E5)
	
	(A6) edge (B6)
	(A6) edge (C6)
	(B6) edge (D6)
	(C6) edge (D6)
	
	%\node[main node,label=above: \large $a$] (A2) at (2,14) {};
	

	;
	\end{tikzpicture}
	\caption{A multiset of 34 behavior graphs representing the traces from an uncertain event log.}
	\label{fig: combi estimates}
\end{figure}
%
%
%
%
%
%
%
%
%
%
\begin{table}[h]
	\centering
	\begin{tabular}{ccc}
		\textbf{Activity trace} & \textbf{$ ~~ \textbf{p}_{unc} ~~$} & \textbf{$~~ \textbf{p}_{log} ~~$}
		\\ \hline
	%\multicolumn{1}{c}{\cellcolor{black!30}\textbf{Case ID}} & 
  %\multicolumn{1}{c}{\cellcolor{black!30}\textbf{Event ID}} &
  %\multicolumn{1}{c}{\cellcolor{black!30}\textbf{Timestamp}}
  %\\\hline
	\multicolumn{1}{|c|}{$\langle a,b,d,e \rangle$} & 
	\multicolumn{1}{|c|}{$0.15$} & 
	\multicolumn{1}{|c|}{$22/41$} 
\\ \hline
	\multicolumn{1}{|c|}{$\langle a,c,d,e \rangle$} & 
	\multicolumn{1}{|c|}{$0.35$} & 
	\multicolumn{1}{|c|}{$12/41 $}
\\ \hline
	\multicolumn{1}{|c|}{$\langle a,d,b,e \rangle$} & 
	\multicolumn{1}{|c|}{$0.15$} & 
	\multicolumn{1}{|c|}{$0$} 
\\ \hline
	\multicolumn{1}{|c|}{$\langle a,d,c,e \rangle$} & 
	\multicolumn{1}{|c|}{$0.35$} & 
	\multicolumn{1}{|c|}{$7/41 $} 
\\ \hline
	
	\end{tabular}
	\caption{The set of activity trace realizations for the uncertain process instance whose behavior graph is the last graph in Fig. \ref{fig: combi estimates}.}
	\label{table: combi traces}
\end{table} 
% 
%
%
%
%
%
%
%
\begin{table}[h]

	\centering
	\begin{tabular}{cccc}
		\textbf{Activity pair $(a_1,a_2)$} & \textbf{$P(a_1,a_2)$} & \textbf{Activity pair $(a_1,a_2)$} & \textbf{$P(a_1,a_2)$}
		\\ \hline
	%\multicolumn{1}{c}{\cellcolor{black!30}\textbf{Case ID}} & 
  %\multicolumn{1}{c}{\cellcolor{black!30}\textbf{Event ID}} &
  %\multicolumn{1}{c}{\cellcolor{black!30}\textbf{Timestamp}}
  %\\\hline
	\multicolumn{1}{|c|}{$(a,b)$} & 
	\multicolumn{1}{|c|}{$\frac{12+10+1+3}{12+10+1+3} = 1$} & 
	\multicolumn{1}{|c|}{$(a,c)$} &
	\multicolumn{1}{|c|}{$\frac{12+10+7+3}{12+10+7+3} = 1$}
\\ \hline
	\multicolumn{1}{|c|}{$(a,d)$} & 
	\multicolumn{1}{|c|}{$\frac{12+7+1+3}{12+7+1+3} = 1$} & 
	\multicolumn{1}{|c|}{$(c,d)$} &
	\multicolumn{1}{|c|}{$\frac{12}{12+7+3} = 6/11$}
\\ \hline
	\multicolumn{1}{|c|}{$(a,e)$} & 
	\multicolumn{1}{|c|}{$\frac{12+10+7+1+3}{12+10+7+1+3} = 1$} & 
	\multicolumn{1}{|c|}{$(c,e)$} &
	\multicolumn{1}{|c|}{$\frac{12+10+7+3}{12+10+7+3} = 1$}
\\ \hline
	\multicolumn{1}{|c|}{$(b,d)$} & 
	\multicolumn{1}{|c|}{$\frac{12+1+3}{12+1+3} = 1$} & 
	\multicolumn{1}{|c|}{$(d,b)$} &
	\multicolumn{1}{|c|}{$\frac{0}{12+1+3} = 0$}
\\ \hline
	\multicolumn{1}{|c|}{$(b,e)$} & 
	\multicolumn{1}{|c|}{$\frac{12+10+1+3}{12+10+1+3} = 1$} & 
	\multicolumn{1}{|c|}{$(d,c)$} &
	\multicolumn{1}{|c|}{$\frac{7}{12+7+3} = 7/22$}
\\ \hline
	\multicolumn{1}{|c|}{$(d,e)$} & 
	\multicolumn{1}{|c|}{$\frac{12+7+1+3}{12+7+1+3} = 1$} & 
	\multicolumn{1}{|c|}{} &
	\multicolumn{1}{|c|}{}
\\ \hline


	\end{tabular}
	\caption{Each activity pair ordering that appears in one of the traces from Table \ref{table: combi traces}, together with its probability computed using the weak-order model.}
	\label{table: combi pairs}
\end{table} 
% 
%
%
\newline
By taking a weighted sum of the two estimates for each activity trace from Table \ref{table: combi traces}, one can obtain a new probability estimate.
For example, picking $w_{log} = w_{unc} = 0.5$, the most likely trace would be $\langle a,b,d,e \rangle$ with $\textbf{p}_{combi}(\langle a,b,d,e \rangle) = 0.5 \cdot 0.15 + 0.5 \cdot 22/41 = 0.34329$.
The least likely trace, on the other hand, would be $\langle a,d,b,e \rangle$ with $\textbf{p}_{combi}(\langle a,d,b,e \rangle) = 0.5 \cdot 0.15 + 0.5 \cdot 0 = 0.075$.

In summary, we proposed a method for combining the probability estimates for the trace realizations of uncertain process instances which exploit both explicit description of uncertainty, and certain trace patterns that appear in the log.
Since the new probability $\textbf{p}_{combi}$ combines the two independent estimates $\textbf{p}_{unc}$ and $\textbf{p}_{log}$ through a weighted sum, the weights $w_{unc}$ and $w_{log}$ can be adjusted, so that for a particular process instance $c$ in an uncertain log $L$, $w_{unc}$ reflects our reliability in the local uncertainty information enclosing the events of $c$, whereas $w_{log}$ depends on the amount and quality of the recorded data in $L$ which displays no uncertainty.
